\section{Einführung neuer Features bei SAP Analytics Cloud}\label{sec:feature}


\subsection{Quellen der Ideen}
\begin{itemize}
    \item Roadmap
    \item Kunden
    \item Andere Teams 
    \item Board
\end{itemize}

Damit eine Idee den Status erreicht, dass die umgesetzt werden kann, müssen verschiedene Konzepte bereits durchdacht und abgestimmt werden. 
\begin{enumerate}
    \item Fromulieren des Features in einer \textit{User-Story}
    \item Ausarbeitung von Architekturkonzept und Testkonzept
    \item UI-Mock
\end{enumerate}
Auf der Grundlage dieser Dokumente entscheiden die Manager in einem \textit{Build-Meeting} anhand von Kriterien ob diese Idee beziehungsweise ob dieses Feature 
umgesetzt wird. Es kann durchaus vorkommen, dass bereits beim Erstellen dieser Konzepte eine Idee abgelehnt wird. In den folgenden Abschnitten werden
die einzelnen Dokumente und Bewertungsprozesse beschrieben.

\subsection{Definition of Ready}
Die \textit{Definition of Ready} gibt an, ob eine Idee zur Umsetzung freigegeben wird.


Soll eine neue Funktionalität umgesetzt werden, wird diese an den Entwickler weitergegeben. Dieser 
arbeitet in engem Kontakt mit Designern und Qualitätsverantwortlichen ein Testkonzept und ein Architekturkonzept heraus. 
Anschließend wird die Idee nochmals bewertet und entweder weiterverfolgt oder abgebrochen. Das Architekturkonzept beschreibt den 
grundlegenden Aufbau des Features. Hierbei wird besonders auf Anpassungen im Backend und der Datenbank wertgelegt. Denn sobald eine Strukturänderung 
an einer Datenbanktabelle notwendig ist, muss berücksichtigt werden, wie eine Migration der Kunden funktioniert. Hierbei dürfen die Tests nicht 
vernachlässigt werden. Sollte es bei einer Migration zu Problemen kommen, könnten Kundendaten beschädigt werden. Im Architekturkonzept wird 
neben dem Backend auch das Frontend betrachtet. Dabei wird beschrieben, ob es zum Beispiel lediglich Anpassungen an einem bereits existierenden
Dialog gibt oder ein vollständig neuer Dialog erzeugt werden muss. 
Das Testkonzept enthält alle Tests die nach der Entwicklung vorgenommen werden. Dabei ist besonders die Kombination mit anderen Entwicklungen 
wichtig. Für jedes Feature müssen automatisierte Integrationstest und Unittests vom Entwickler programmiert werden. Zusätzlich werden verschiedene
manuelle Tests aufgelistet und gegebenenfalls Migrationstest.\\
Erst wenn diese beiden Konzepte von Entwickler, Designer und Qualitätsverantwortlichen formuliert wurden und im Build-Meeting von den Managern abgesegnet 
wurden, beginnt die Entwicklung des Features. 
\begin{itemize}
    \item User-Story-Labels: Innovation, Customer, Improvement, Escalation
    \item Architecture APproved
    \item Test strategie ready 
    \item deployment impact 
\end{itemize}
\subsection{Ideenbewertung: Build-Meeting}
\begin{itemize}
    \item Manager sitzen zusammen 
    \item Review vom Feature
\end{itemize}
\subsection{Umsetzungsentscheidungen}
Auch während der Umsetzung arbeiten Designer, Qualitätsverantwortliche und Entwickler 
\begin{itemize}
    \item Zusammenarbeit mit Designern
    \item Zusammenarbeit verschiedener Teams, Verantwortlichkeit
    \item QUalitätsanspruch: Testing! 
    \item \ac{arc} - Definition of Done
\end{itemize}