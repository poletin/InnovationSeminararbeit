%!TEX root = ../seminararbeit.tex
\section{Einführung neuer Features bei \ac{sac}}\label{sec:feature}

\subsection{Quellen der Ideen}
Die Ideen für die Funktionalitäten der Anwendung stammen aus unterschiedlichen Quellen. Die wichtigste Quelle ist hierbei der Kunde selbst. 
Einige spezielle Kunden arbeiten eng mit Beratern von \ac{sac} zusammen. Diese Berater werden im Scrum-Umfeld als \textit{Product-Owner} bezeichnet. 
Kundenanfragen werden über die Berater an die Manager der Teams weitergetragen. Neben dem direkten Feedback der Kunden gibt es ein 
Fehler-Management-System, in dem Kunden Probleme anlegen können.\\
Eine weitere Quelle ist der Abteilungsleiter und die Vorstände von SAP. Zu Beginn eines neuen Jahres wird von allen Managern von \ac{sac}
eine \textit{Roadmap} zusammengestellt. Dies Roadmap ist allerdings nicht statisch, sie wird über das Jahr hinweg mehrmals angepasst. 
Es gibt eine Gesamt-Roadmap in der Themen beschrieben werden, die die gesamte Abteilung betroffen. Ein aktuelles Thema ist beispielsweise
die Performance der Anwendung. Für jedes Team gibt es eine weitere individualisierte Roadmap, im Modeling-Team ist ein aktuelles Thema beispielsweise 
eine Neuentwicklung der Oberfläche.\\
Es können auch interne Ideen für neue Funktionalitäten bereitgestellt werden, zum Beispiel von Mitarbeitern einzelner Teams.
Diese werden direkt über die Manager kommuniziert.\\
Für \ac{sac} gibt es ein Team, das basierend auf \ac{sac} eine \textit{Boardroom-Anwendung} entwickelt, die speziell für die Vorstände von SAP angepasst ist.
Deshalb sind auch die Vorstandsmitglieder aktive Nutzer des Produkts und stellen Anfragen für Funktionalität.\\
Die Ideen aus verschiedenen Quellen werden in einem Ideenprozess verarbeitet. Dabei wird entschieden, ob die Ideen umgesetzt werden oder nicht. Dieser Prozess
wird im folgenden Abschnitt vorgestellt.

\subsection{Ideenvorverarbeitung}
Damit eine Idee den Status erreicht, dass die umgesetzt werden kann, müssen verschiedene Konzepte bereits durchdacht und abgestimmt werden. 
\begin{enumerate}
    \item Fromulieren des Features in einer \textit{User-Story}
    \item Ausarbeitung von Architekturkonzept und Testkonzept
    \item UI-Mock
\end{enumerate}
Auf der Grundlage dieser Dokumente entscheiden die Manager in einem \textit{Build-Meeting} ob dieses Feature 
umgesetzt wird. Es kann durchaus vorkommen, dass bereits beim Erstellen dieser Konzepte eine Idee abgelehnt wird. In den folgenden Abschnitten werden
die einzelnen Dokumente und Bewertungsprozesse beschrieben.

Die \textit{Definition of Ready} gibt an, ob eine Idee zur Bewertung freigegeben wird.
Soll eine neue Idee umgesetzt werden, wird diese an den Entwickler weitergegeben. Dieser 
arbeitet in engem Kontakt mit Designern und Qualitätsverantwortlichen ein Testkonzept und ein Architekturkonzept heraus. 
Das Architekturkonzept beschreibt den grundlegenden Aufbau des Features. 
Das Testkonzept enthält alle Tests die nach der Entwicklung vorgenommen werden. Dabei ist besonders die Kombination mit anderen Funktionalitäten 
wichtig. Für jedes Feature müssen automatisierte Integrationstest und Unittests vom Entwickler programmiert werden. Zusätzlich werden verschiedene
manuelle Tests aufgelistet.\\
Erst wenn diese beiden Konzepte von Entwickler, Designer und Qualitätsverantwortlichen formuliert wurden, wird diese im 
\textit{Build-Meeting} von den Managern bewertet und abgesegnet. Diese Vorverarbeitung der Ideen sichert die Machbarkeit bei der Umsetzung. Zusätzlich ergibt 
sich eine Aufwandsabschätzung darüber, wie viel Aufwand für die Entwicklung einzuplanen ist. 

\subsection{Ideenbewertung: Build-Meeting}
Das Build-Meeting ist ein Treffen aller Teammanager. Jeder Manager hat die Möglichkeit Ideen und Konzepte für zukünftige Entwicklungen 
vorzustellen. 

\begin{quote}Anhand von klar formulierten Kriterien werden die Ideen bewertet. So wird darüber entschieden welche Ideen umgesetzt werden oder nicht.\end{quote}

Leider zeigt sich in der Praxis, dass die Wirklichkeit, besonders bei der Weiterentwicklung großer Produkte, eine andere ist. 
In den Build-Meetings werden die beiden Konzepte betrachtet. Sind diese schlüssig wird eine Idee zur Umsetzung frei gegeben. 
Das Problem in der Praxis ist, dass hinter den Ideen oft Kunden stehen, die eine weitere Funktionalität fordern. Es obligt nicht 
dem Entwicklunsteam Ideen nach Fragen wie: 
\begin{itemize}
    \item Passt die Umsetzung dieser Idee in das Gesamtprodukt?
    \item Bringt es allen Kunden einen Mehrwert oder ist es eine spezifische Lösung für einen einzelnen Kunden?
    \item Hebt sich der Aufwand für die Umsetzung durch den Mehrwert für den Kunden auf?
\end{itemize}
zur beurteilen. Es sind eher Fragen wie: 
\begin{itemize}
    \item Wie kann die Idee in das Produkt integriert werden?
    \item Wie viel Zeit muss für die Umsetzung geplant werden? 
    \item Welche Teams sind an der Umsetzung dieser Idee beteiligt?
\end{itemize}
die beantwortet werden müssen. 
Dies zeigt, dass der theoretische Prozess, Kriterien formulieren und anwenden, in der Praxis nicht immer eingehalten werden kann.
Die Bewertung der Ideen ist im Entwicklungsprozess bei \ac{sac} implizit in den einzelnen Vorverarbeitungsschritten einer Idee vorhanden.
Sobald ein sinvolles Konzept für eine Idee erstellt werden konnte, kann diese eingeplant und implementiert werden.