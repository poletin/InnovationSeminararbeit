\section{Theorie: Kriterien zur Ideenbewertung}\label{sec:kaptiel}
\todo[inline]{Kapiteleinleitung}

\subsection{Ideenbewertung}
\begin{itemize}
    \item Warum sind die Kriterien zur Ideenbewertung so wichtig?
    \item Allgemeine Kriterien
    \item aufgabenspezifische Kriterien
\end{itemize}
\subsection{Randbedingungen vs Erfolgskriterien}
\textbf{Randbedingungen} beschreiben die Eigenschaften, die eine Idee haben beziehungsweise
nicht haben soll. Sie sind absolut und werden oft als \textit{Muss-Kriterien}
und \textit{Darf-Nicht-Kriterien} bezeichnet.
Typische Randbedingungen sind beispielsweise die \textit{Kosten}. 
Es wird festgelegt welchen Betrag eine Ideenumsetzung maximal kosten darf. 
Weitere typische Randbedingungen sind \textit{Fit}, die Idee muss zum Unternehmen 
beziehungsweise zum Image des Unternehmens passen, oder \textit{Ressourcen}, die Ressourcen für die
Umsetzung müssen vorhanden sein. 
Randbedingungen sind entweder erfüllt oder nicht. Je nachdem wird eine Idee verworfen oder 
weiterverfolgt.


\subsection{Kriterien formulieren}