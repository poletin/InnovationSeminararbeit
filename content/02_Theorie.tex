%!TEX root = ../seminararbeit.tex
\newpage
\section{Theorie: Ideenbewertung}\label{sec:theorie}
Ideen zu bewerten ist ein wichtiger Schritt in der Entwicklung neuer Produkte. 
Auch während der Weiterentwicklung bereits existierender Lösungen darf eine Bewertung im Entwicklungsprozess nicht fehlen.\\
Durch Ideenfindungsmethoden, wie zum Beispiel Brainstorming, werden sehr viele Ideen generiert. Doch diese Ideen 
sind nicht alle zielführend. 
Durch Bewertung und Kategorisierung können vielversprechende Ideen aus einer großen Menge heraus gefiltert werden. 
Sich auf zielführende Ideen zu fokussieren, hat besonders wirtschaftliche Gründe, denn je mehr Zeit vergeht, bis eine Idee 
als nicht umsetzbar oder nicht zielführend erkannt wird, desto mehr Kosten entstehen für das Unternehmen.
Ziel der Ideenbewertung ist es, Ideen frühzeitig zu filtern, um so das Risiko misslungener Investitionen zu vermeiden.
\cite{grossklaus:2008}\\
\autoref{img:filterKosten} zeigt eine Statistik, welche die steigenden Kosten bei fortschreitender Entwicklung zeigt. Die
Statistik zeigt auch, dass sich die Anzahl der Ideen verringert, jedoch die Kosten pro Idee steigen.
\begin{figure}[h]
	\centering
	\includegraphics[width=10cm]{ideenfilterung.png}
	\caption{Ausscheidungsquoten und Kostenentwicklung}
	\label{img:filterKosten}
\end{figure}

\subsection{Kriterien zur Ideenbewertung}
Der Kern einer Ideenbewertung sind die Kriterien, anhand derer über eine Idee geurteilt wird.
Aus diesem Grund ist es wichtig, diese sorgfältig zu definieren. Werden die Kriterien falsch oder nicht sorgfältig gewählt,
so kann dies zu Ablehnungs- oder Annahmefehlern führen.\\ 
Ablehnungsfehler bezeichnen Ideen, die fälschlicherweise abgelehnt wurden, obwohl sie bei genauerer Betrachtung 
zielführend gewesen wären. Das Gegenteil sind Annahmefehler. Dabei wird eine Idee weiterverfolgt, obwohl diese nicht zum Ziel führt. 
Während Ablehnungsfehler zu einer Verzögerung in der Entwicklung führen, verursachen Annahmefehler verschwendete Zeit und vergeudetes Geld.
Kriterien lassen sich grundsätzlich in zwei Kategorien unterteilen. Allgemeine und aufgabenspezifische Kriterien.\\
\paragraph{Allgemeine Kriterien} lassen sich auf verschiedene Ideen und Anwendungsfälle übertragen.
Um diese Kriterien zu definieren, ist es nicht notwendig, die konkrete Aufgabe zu kennen. 
Die häufigsten allgemeinen Kriterien sind \textit{Attraktivität}, \textit{Realisierbarkeit} und 
\textit{Disruptionspotential}. Hierbei wird deutlich, dass es sich um grundsätzliche Kriterien handelt, 
welche unabhängig von einer bestimmten Thematik sind.\\
\paragraph{Aufgabenspezifische Kriterien} hingegen erfordern genaue Kenntnisse der Aufgabenstellung. Es muss klar sein, welches 
konkrete Kundenbedürfnis mit einer Idee befriedigt werden soll. Auch verschiedene Markttrends müssen hierbei beachtet werden. 
Diese haben oft entscheidenden Einfluss auf die Erfolgschancen einer Idee. Daher müssen solche Kriterien für jede Aufgabe und jedes Ziel 
individuell formuliert werden.
\cite{grossklaus:2008}

\subsection{Kriterien der Ideenbewertung nach Zephram}
Auch Zephram \cite{zephram:2018} teilt die Kriterien in zwei Kategorien ein. Allerdings setzt er einen anderen Fokus und verfolgt 
damit ein anderes Ziel.

\paragraph{Randbedingungen} beschreiben die Eigenschaften, die eine Idee haben muss - 
beziehungsweise nicht haben darf. Sie sind absolut und werden als \textit{Muss-Kriterien}
und \textit{Darf-Nicht-Kriterien} bezeichnet.
Typische Randbedingungen sind beispielsweise die \textit{Kosten}. 
Es wird festgelegt welchen Betrag eine Ideenumsetzung maximal kosten darf. 
Eine weitere typische Randbedingung ist \textit{Fit}. Die Idee muss zum Unternehmensimage passen. Zum anderen 
gibt es die \textit{Ressourcen}. Die Bestandteile für eine Umsetzung müssen vorhanden sein. 
Randbedingungen sind entweder erfüllt oder nicht erfüllt. Je nachdem wird eine Idee verworfen oder weiterverfolgt. 

\paragraph{Erfolgskriterien} sind nicht absolut. Sie beschreiben Eigenschaften einer Idee, 
mit denen diese als erfolgreich gilt. Ziel ist es nicht, Ideen auszuschließen, sondern die
vielversprechendste Idee herauszufiltern. Das heißt im Umkehrschluss, alle anderen Ideen werden nicht 
verworfen, sondern zunächst nicht ausgewählt. Nicht selten werden die nicht ausgewählten Ideen aufbewahrt, da diese zu späterem 
Zeitpunkt wiederverwendet werden können. Erfolgskriterien werden als \textit{Soll-Kriterien} bezeichnet. Um 
Erfolgskriterien einheitlich zu formulieren kann der Satz \textit{"Je mehr..., desto besser"} verwendet werden.
Er verdeutlicht, dass sich Erfolgskriterien nicht in \textit{erfüllt} und \textit{nicht erfüllt} aufteilen 
lassen. Typische Erfolgskriterien sind \textit{Gewinnpotential}, \textit{Wachstumspotential} oder \textit{Kundennutzen}.

\paragraph{Kann-Kriterien} sollten bei der Bewertung von Ideen keine Rolle spielen und wurden aus diesem Grund bisher nicht erwähnt.
Es handelt sich hierbei um Kriterien, die nicht zwingend notwendig für den Erfolg einer Idee sind. Wird eine Idee anhand 
dieser Kriterien bewertet, so sind die Erfolgskriterien nicht gut gewählt und sollten demnach angepasst werden.\\

Beim \textbf{Formulieren} der Randbedingungen ist es wichtig, dass sich die Kriterien nicht
widersprechen. Die Folge wäre, dass alle Ideen aussortiert werden, da niemals alle Randbedingungen erfüllt sein können.
Als häufiges Beispiel ist hier die Forderung nach einer Systemausfallsicherheit von 99\%  bei einer
kurzen Entwicklungszeit. 
Es kommt bei den Kriterien darauf an, das richtige Mittelmaß für die Aufgabe zu finden. 
Sowohl Randbedingungen wie auch Erfolgskriterien sollten zur Aufgabe aber auch zum Unternehmen passen. Sie können sowohl aufgabenspezifisch
als auch allgemein sein. Eine weitere Schwierigkeit ist die möglichst konkrete
Formulierung der Kriterien, um diese möglichst effizient anwenden zu können. \\

Beim \textbf{Anwenden} der Kriterien zur Bewertung von Ideen müssen zunächst die Randbedingungen angewandt werden. 
Dies sorgt bereits für eine starke Reduktion der Ideenanzahl. \autoref{img:filterKosten} zeigte dies in einer Statistik. 
Um hierbei keine Fehler zu machen, wird ein Vier-Augen-Prinzip empfohlen. 
Schwieriger ist die anschließende Anwendung der Erfolgskriterien, da diese graduell sind, allerdings keine klar 
definierte Messskala besitzen. 
Es gibt einige Methoden, die die Arbeit mit Erfolgskriterien vereinfachen. Einige werden 
in der folgenden Liste aufgezählt: 
\begin{itemize}
    \item Punktekleben (Siehe in Kapitel \ref{sec:retro-punkte})
    \item Nutzwertanalyse
    \item Paarvergleichsmatrix
\end{itemize}
Die Methoden die in der Abteilung \ac{sac} angewendet werden, werden im Praxisteil genauer beschrieben. \cite{zephram:2018}

\subsection{Vorgehensweise nach Schawel und Billing}
Der Vorschlag von Schawel und Billing ist es, die Ideenbewertung in zwei Phasen zu unterteilen. Zunächst muss die gesamte Anzahl 
der Ideen verringert werden. Dies geschieht in der \textbf{Ideenkategorisierung}. Hierbei geht es darum, Ideen unter 
Oberbegriffen zusammenzufassen. Dies hilft vorallem dabei, ähnliche Ideen oder Überlappungen frühzeitig zu erkennen.
Zusätzlich hat diese Vorgehensweise den Vorteil, dass die gefundenen Kategorien später in Arbeitspakete überführt werden 
könnten. Anschließend können die Kategorien im Schritt der \textbf{Ideenbewertung} genauer analysiert werden. 
Dies erfolgt anhand formulierter Kriterien. Für eine erste Einschätzung kann außerdem eine zweidimensionsale-Matrix verwendet werden, 
welche die beiden Achsen \textit{Wirkung} und \textit{Realisierbarkeit} entgegenstellen. Ideen, die eine geringe Wirkung haben und/oder 
schwer zu realisieren sind, können so leicht identifiziert werden. \cite{schawel:2009}