\section{Fazit}\label{sec:fazit}
Die in dieser Seminararbeit beschriebenen Verfahren und Konzepte zu Kriterien der Ideenbewertung zeigen, dass
es nicht möglich ist, Kriterien einmal zu definieren und für viele Innovationen zu verwenden. Die Formulierung der Kriterien
zur Bewertung von Ideen ist sehr aufwändig und zeitintensiv. Denn diese müssen sehr sorgfältig gewählt, definiert und beschrieben 
werden. Dieser Aufwand wird durch das sinkende Risiko von Fehlinvestitionen kompensiert.\\ 
Je nachdem wie groß die Risikobereitschaft ist, sollte die Ideenbewertung sorgfältig durchgeführt werden. 
Doch es ist Vorsicht geboten: Keine Idee kann durch Vorüberlegungen zu 100\% zielführend sein. Das Risiko von 
Fehlinvestitionen verschwindet nicht vollständig. Deshalb sollte zwischen Detaillierungsgrad der Ideenbewertung 
und Risikobereitschaft abgewogen werden. Wird zu viel Zeit in der Vorarbeit einer Entwicklung investiert, kann es sein, dass
der richtige Zeitpunkt für die Markeinführung verpasst wird. Wie komplex es sein kann, Kriterien für große Produkte zu formulieren 
zeigt \autoref{sec:feature-kriterien}.\\
Kern der Ideenbewertung ist es, sich vorab mit einer Idee zu beschäftigen und Schwachstellen zu finden. Dies 
kann auch mit anderen Methoden umgesetzt werden wie in \autoref{sec:feature} beschrieben. Es kann sehr schwierig sein, 
eine Ideenbewertung in einem bereits existierenden und komplexen Prozess neu zu definieren. Durch Konzeptarbeiten kann 
das Risiko, dass eine Funktionalität nicht umgesetzt werden kann minimiert werden, da auch hier vorab eine Idee genauer 
analysiert wird. \\
Voraussetzung und damit zusätzlich erschwerend, ist es, dass Kriterien nur angewendet werden können, wenn diese abschätzbar sind. 
Das heißt beispielsweise, es muss vor der Entwicklung überschaubar sein, wie zeit- und kostenintensiv eine Entwicklung wird. 
In kleinen Prozessen wie beispielsweise in \autoref{sec:retro} beschrieben können die Verfahren zur Ideenbewertung sehr leicht und 
zielführend eingesetzt werden, da hier die Kriterien leichter zu finden sind und das Anwenden unkomplizierter ist. \\
Insgesamt sollte der Ideenbewertungsprozess bei neuen Entwicklungen nicht vernachlässigt werden. Es muss klar definiert sein, wie
Ideen gefunden und weiter entwickelt werden. Sei es mit der Definition von Kriterien oder mithilfe anderer Verfahren. 